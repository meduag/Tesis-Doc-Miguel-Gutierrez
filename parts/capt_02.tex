\chapter{FUNDAMENTOS TEÓRICOS}
\label{sec:cap2}

%\resumodocapitulo{Colocar aqui un resumen del capitulo}

\section{SISTEMA NEUROMUSCULAR}

O sistema neuromuscular é composto pelos sistemas nervoso e muscular intrinsecamente ligados e sendo estes os responsáveis por inúmeras funções do corpo humano, desde coordenar uma grande quantidade de músculos para permitir a locomoção até sincronizar controladamente os músculos do coração e garantir a circulação sanguínea \cite{Moore2014AnatomiaClinica}. A seguir é fornecida uma breve descrição do sistema nervoso e seguidamente do sistema muscular.

\subsection{O SISTEMA NERVOSO}

O sistema nervoso está dividido em duas partes: o Sistema Nervoso Central (\acrshort{SNC}) e o Sistema Nervoso Periférico (\acrshort{SNP}). Juntos o \acrshort{SNC} e o \acrshort{SNP} controlam três funções principais: entrada sensorial, integração de informações e ações voluntárias e involuntárias, incluindo a resposta motora \cite{Benjamin1978EncyclopediaHealth}. O \acrshort{SNC}, composto pelo cérebro e medula espinhal, integra e processa a informação enviada pelos nervos. O \acrshort{SNP} inclui nervos que carregam mensagens sensoriais para o \acrshort{SNC} e nervos que enviam informações do \acrshort{SNC} para os músculos e glândulas \cite{Gerard2010TheSystem}. O \acrshort{SNP} é ainda dividido em sistema somático e sistema autônomo. O sistema somático é formado por receptores sensoriais na cabeça e nas extremidades, nervos que transportam informação sensorial para o \acrshort{SNC} e nervos que carregam instruções do \acrshort{SNC} para os músculos esqueléticos. O sistema autonômico controla as secreções glandulares e o funcionamento de músculos lisos e cardíacos \cite{Mai2012TheSystem}. As divisões simpáticas e parassimpáticas do sistema autônomo muitas vezes trabalham em oposição uma à outra para regular os processos involuntários do corpo. Processos involuntários, como batimentos cardíacos e peristaltismo, são aqueles que não exigem ou envolvem controle consciente \cite{Gerard2010TheSystem}. A Figura \ref{fig:ogsn_f1} mostra como está organizado o sistema nervoso. 

\begin{figure*}
    \centering %
    \small %
    %\includegraphics[width=\linewidth]{figs/Fig_c2/ogsn.pdf}
    \def\svgwidth{0.9\columnwidth}% Código LATEX define o tamanho da figura 
    \import{figs/Fig_c2/}{ogsn2.pdf_tex}
    \caption{Organização geral do sistema nervoso.}
    \label{fig:ogsn_f1}
\end{figure*}

O sistema nervoso é composto por neurônios e células gliais (ver Figura \ref{fig:nns_f2}). Os neurônios são a estrutura básica e unidade funcional do sistema nervoso suportadas pelas células gliais. Os neurônios são unidades funcionais de informação, operando em grandes conjuntos para transmitir e processar informações do sistema nervoso. Os neurônios podem se dividir em três partes: o corpo celular, que é a parte mais volumosa da célula nervosa, os dendritos, que são prolongamentos finos e ramificados que levam os estímulos captados do ambiente ou de outras células até o corpo celular e, o axônio, caracterizado por um prolongamento fino e mais longo que os dendritos (e.g. humanos podem possuir axônios que se estendem por mais de um metro), cuja função é transmitir para as outras células os impulsos nervosos provenientes do corpo celular \cite{Lent2002CemNeuronios}.


\begin{figure*}
    \centering %
    \small %
    %\includegraphics[width=\linewidth]{figs/Fig_c2/ogsn.pdf}
    \def\svgwidth{1\columnwidth}% Código LATEX define o tamanho da figura 
    \import{figs/Fig_c2/}{nervo_neuronio_sinap.pdf_tex}
    \caption{Ilustração de um nervo, neurônio e sinapses (adaptado de \cite{Fonseca2015InstrumentacaoMedular}).}
    \label{fig:nns_f2}
\end{figure*}

A comunicação dos estímulos, comumente chamada de sinapses nervosa, pode-se dar entre neurônios, neurônios-músculos e neurônio-glândulas. A sinapse nervosa é de origem eletroquímica e permite transmitir, inibir ou modificar mensagens. As mensagens são transmitidas por impulsos nervosos na membrana plasmática das células e permite que seja criado e transmitido um impulso elétrico como meio de propagação da informação transmitida pelo axônio \cite{Lent2002CemNeuronios}.

A membrana do neurônio pode possuir dois estados, repouso e despolarização. Quando a membrana está em repouso, o seu lado externo possui uma carga positiva e o interno uma carga negativa, este estado se conhece como membrana polarizada e caracteriza um potencial de repouso. Quando há um estímulo elétrico na membrana do neurônio, a permeabilidade da membrana é alterada ocorrendo troca de íons, o que induz uma despolarização, transformando o potencial de repouso em potencial de ação. É importante ressaltar que, para que ocorra a despolarização da membrana e se produza o potencial de ação, o estímulo elétrico tem que ultrapassar um limiar para desencadear e propagar o impulso nervoso \cite{Lent2002CemNeuronios}.  

Dependendo do local do impulso, este pode ser capturado por técnicas de registro de biopotenciais (sinais elétricos produzidos por seres vivos) como eletromiografia (\acrshort{EMG}) para sinais dos músculos estriados esqueléticos, eletrocardiografia (\acrshort{ECG}) para sinais cardíacos, entre outros \cite{Fonseca2015InstrumentacaoMedular}.

\subsection{O SISTEMA MUSCULAR}

O sistema muscular humano é composto de 600 músculos, que representam dentre 40\% a 50\% da massa corporal de uma pessoa. Cada unidade muscular tem as propriedades de se contrair e relaxar, e está relaciona a todas as ações físico-mecânicas do corpo humano, como caminhar, correr, comer, deglutir, manter a circulação do sangue e respirar. Todas essas tarefas realizadas pelos músculos também dependem do controle e regulação do sistema nervoso \cite{Vanputte2016AnatomiaSeeley}.

De forma geral, o sistema muscular humano pode ser identificado por três tipos de músculos: o liso ou não estriado, o estriado cardíaco e o estriado esquelético. Os músculos lisos são aqueles que têm contrações lentas e produzidas de forma involuntária. São responsáveis por funções como o movimento dos órgãos internos, veias e artérias, entre outros. O músculo cardíaco estriado está relacionado com miocárdio, que possui um funcionamento autônomo e involuntário, executor da circulação sanguínea. Finalmente, o músculo estriado esquelético é aquele que se fixa nos ossos por meio de tendões, por exemplo os músculos das pernas, cuja função principal é a locomoção. A contração do músculo estriado é, então, comandado de forma voluntária nas mais diversas funções \cite{Geraldo2007AnatomiaSegmentar}. 

\subsection*{A unidade motora}

O tecido muscular permite por meio da contração converter energia química em trabalho mecânico por meio da contração. A contração muscular é controlada por neurônios motores ou motoneurônios que em conjunto com a fibra muscular esquelética, constitui uma unidade funcional ou unidade motora. Qualquer tipo de atividade motora supõe em último termo a necessidade de que os motoneurônios disparem potenciais de ação, sendo que, em caso de desnervação por qualquer que seja o motivo, levará a um estado de paralisia muscular \cite{Jones2004SkeletalMuscle}. Aqui vale a ressalva que, estimulando artificialmente um motoneurônio por meio de corrente elétrica e alcançando os níveis de despolarização, pode-se imitar o comando do sistema nervoso, ocasionando uma contração das fibras musculares que o neurônio motor inerva \cite{Lent2002CemNeuronios}.

Sempre que o motoneurônio gera um potencial de ação, as fibras musculares correspondentes se contraem, indicando uma dependência do músculo esquelético ao motoneurônio de natureza não apenas estrutural, mas se entendendo para o campo funcional, o que é evidenciado, no caso da desenervação, pelos casos em que as fibras musculares esqueléticas deixam de se contrair e se atrofiam \cite{JoseLopezChicharro2006FisiologiaEjercicio}. Na Figura \ref{fig:um_f3} é apresentado o conjunto fibra muscular e motoneurônio que constituem a unidade motora, conformando, então, uma unidade estrutural e funcional.

\begin{figure*}[h]
    \centering %
    \small %
    %\includegraphics[width=\linewidth]{figs/Fig_c2/ogsn.pdf}
    \def\svgwidth{1\columnwidth}% Código LATEX define o tamanho da figura 
    \import{figs/Fig_c2/}{um_f3.pdf_tex}
    \caption{Unidades motoras (adaptado de \cite{JoseLopezChicharro2006FisiologiaEjercicio}).}
    \label{fig:um_f3}
\end{figure*}

\section{ESTIMULAÇÃO ELÉTRICA}

Por definição, a estimulação elétrica é um procedimento no qual é aplicada corrente elétrica para estimular artificialmente o conjunto de tecido neural e/ou neuromuscular com propósitos terapêuticos \cite{Benjamin1978EncyclopediaHealth}. Desde a sua origem, a estimulação elétrica tem sido usada para o tratamento de diversas patologias do sistema neuromuscular \cite{Erick2002EstimulacionUrologia}. A seguir são descritos alguns fatos relevantes sobre a história da estimulação elétrica, a sua aplicação e classificação segundo a corrente de estimulação, uso em terapia e, por fim, as suas vantagens e desvantagens.

\subsection{ELEMENTOS HISTÓRICOS}
O uso da estimulação elétrica remete à antiguidade. Na época, os antigos egípcios e depois os gregos e romanos descobriram que os peixes elétricos (e.g. enguias) eram capazes de gerar “choques elétricos” (i.e. estímulo elétrico), o que foi utilizado para o alívio da dor. Mais tarde, nos séculos 18 e 19, esses animais foram trocados por dispositivos de estimulação elétrica criados pelo homem. Isto aconteceu em várias etapas \cite{Heidland2013NeuromuscularWaisting}, como ilustrado no Quadro \ref{q1}. 

\smallskip

\begin{quadro}
    \caption{Etapas históricas da estimulação elétrica.}
    \centering
    \begin{tabular}{|>{\centering\arraybackslash} m{2.5cm}|p{8cm}|c|}
        \hline
        \rowcolor{lightgray} \textbf{Nome da técnica} &\textbf{Caraterística principal} & \textbf{Ano} \\
        \hline
        %\mbox{%
        %\begin{minipage}[b][0.7cm][c]{2cm}
        Franklinismo
        %\end{minipage}} \hfill  
        & Aplicação de corrente elétrica estática produzida por um gerador de atrito & 1750 \\
        \hline
        Galvanismo & Aplicação de corrente elétrica direta e pulsada à pele por meios químicos  & 1791  \\
        \hline
        Faradismo & Aplicação de corrente elétrica induzida de forma intermitente com polaridade variável & 1838  \\
        \hline
        d'Arsonvalisation  & Aplicação de correntes de alta frequência  & 1881 \\
        \hline     
    \end{tabular}
    \label{q1}
\end{quadro}

Neste contexto, entre os séculos 18 e 19 teve início uma grande divulgação da estimulação elétrica para uso em terapia. À época foi criado o termo eletroterapia, utilizado até hoje na medicina. No entanto, na primeira metade do século 20, pessoas sem formação acadêmica que usavam a estimulação elétrica com fins pouco sérios fizeram com que a eletroterapia fosse ridiculizada ao não possuir uma base científica e evidencia dos seus benefícios. A eletroterapia passou, então, por um ressurgimento na segunda metade do século, quando se intensificaram pesquisas usando, por exemplo, modelos animais. A partir daí o uso de eletroestimulação em investigações clínicas, bem como seus mecanismos neurofisiológicos, foram elucidados detalhadamente, fornecendo uma base científica à eletroterapia nas diversas aplicações da estimulação elétrica \cite{Heidland2013NeuromuscularWaisting}.

Com o tempo, a estimulação elétrica ampliou sua gama de aplicações em diversos contextos, uma vez que a evolução de novas tecnologias, componentes eletrônicos e microcontroladores, permitiu aos engenheiros desenvolver aparelhos mais sofisticados para controlar a corrente de forma mais precisa. Diante disso, surgiu uma grande variedade de equipamentos eletromédicos de estimulação elétrica e em consequência, a concepção de normas de parametrização e segurança dos mesmos \cite{Sanches2013SistemaParaplegicos}.

\subsection{PRINCÍPIOS}

A estimulação elétrica é comumente usada na prática clínica e em laboratórios de pesquisa para tratar e/ou avaliar uma ampla gama de condições que comprometem o sistema neuromuscular, seja relacionado ao sistema somático ou autonômico \cite{Fernando2017ARevisao}. A seguir descreve-se como é entregado o estímulo elétrico ao tecido neural ou ao músculo e as formas de onda usadas na eletroestimulação.

A estimulação elétrica pode ser aplicada usando eletrodos de superfície ou implantados, podendo ser eles metálicos, siliconados, entre outros. O eletrodo implantado permite maior seletividade muscular e por estar em contato direto com o nervo requer menos energia para provocar uma contração. A grande desvantagem do eletrodo implantado reside em ser invasivo, podendo ser tratado como corpo estranho e consequentemente produzindo a possibilidade de rejeição, ou, inclusive, quebrar-se pelo constante uso. Por outro lado, os eletrodos de superfície permitem aplicação mais simples, muito embora a ativação de tecido neuromuscular se dê de forma menos seletiva \cite{Popovic2000ControlDisabled}. De fato, quando os eletrodos de superfície são usados, muitas vezes estruturas nervosas e musculares são estimuladas simultaneamente \cite{Linares2004BibliographicalCuadriceps}. 

Ao usar eletrodos de superfície, estabelecem-se impedâncias relacionadas com a pele e o conjunto eletrodo-pele. Esta impedância influencia diretamente na resposta muscular, assim como a intensidade do impulso utilizado para gerar a contração. Considerando a tecnologia atual de eletrodos de superfície, a impedância eletrodo-pele está documentada na literatura como um valor variável dentre 500$\mathrm{\Omega}$ e 5k$\mathrm{\Omega}$ \cite{Wu2002AApplications}. O valor exato depende de inúmeras variáveis, incluindo condições físicas do paciente. Assim, autores utilizam para fins de pesquisa e projeto de estimuladores um valor de carga em torno de 1k$\mathrm{\Omega}$ \cite{Zanotti2003PeripheralStimulation, Maffiuletti2011ElectricalIssues}. Cabe mencionar que o valor da impedância entre o eletrodo e a pele pode ser diminuído aplicando-se gel condutor entre eles \cite{Faria2006ImplementacaoMedulares}. 

Além da característica anterior, o tamanho e o formato do eletrodo de superfície influenciam na quantidade de corrente que atinge o músculo, e consequentemente a contração muscular. A relação entre o tamanho do eletrodo e a densidade de corrente é inversamente proporcional, ou seja, quanto menor o tamanho do eletrodo, maior densidade de corrente e vice-versa. Neste contexto, tem que se tomar cuidado ao aplicar terapia com ES quando são usados eletrodos de superfície muito pequenos, pois a densidade de corrente pode chegar a níveis que provocariam lesões (sobretudo queimaduras) no paciente \cite{Faria2006ImplementacaoMedulares, Delitto1988ElectricalSurgery}. A Figura \ref{fig:esn_f4} ilustra o estímulo elétrico aplicado por par de eletrodos, bem como a colocação dos eletrodos de superfície e seu funcionamento distribuindo a carga no conjunto pele-músculo.

\begin{figure*}[h]
    \centering %
    \small %
    %\includegraphics[width=\linewidth]{figs/Fig_c2/ogsn.pdf}
    \def\svgwidth{0.9\columnwidth}% Código LATEX define o tamanho da figura 
    \import{figs/Fig_c2/}{esn_f4.pdf_tex}
    \caption{Estimulação elétrica do nervo por meio de eletrodos de superfície (adaptado de  \cite{Robinson2002EletrofisiologiaEletrofisiologico}).}
    \label{fig:esn_f4}
\end{figure*}

\subsection{FORMA DE ONDA}
Podem existir diferentes tipos de terapia com estimulação elétrica variando a forma de onda usada e seus parâmetros. Isto se relaciona com os efeitos buscados na terapia \cite{Paulo2007EfeitosEsqueletico}. Assim, conforme já visto, os parâmetros de estimulação (ver Tabela \ref{tab:pe1}) podem ser condensados na forma de onda.

O sentido da corrente no sinal é conhecido em eletrofisiologia como fase, diferentemente do conceito de fase usado em engenharia \cite{Robinson2002EletrofisiologiaEletrofisiologico}. Por exemplo, um sinal monofásico tem fluxo só em um sentido, seja ele, positivo ou negativo. Existe também a onda ou sinal bifásico, que possui corrente no sentido positivo e negativo alternadamente. Nesse contexto, a Figura \ref{fig:fon_f5} mostra as formas de onda monofásicas mais comuns usadas nas diferentes aplicações da estimulação elétrica. 

Entretanto, os sinais monofásicos são pouco usados na terapia, pois produzem acúmulo de carga elétrica no tecido onde é aplicada a estimulação. Isto é devido à caraterística do sinal que possui uma carga desbalanceada, diferente do sinal bifásico que é balanceado ou simétrico, o que evita o acúmulo de carga no tecido ou no eletrodo \cite{Wu2002AApplications}.

A Figura \ref{fig:onb_f6} ilustra um exemplo do sinal bifásico e seus parâmetros como frequência (que tipicamente varia entre 1 e 100Hz para aplicações em tecido neuromuscular), intensidade/amplitude (a depender da aplicação e do eletrodo utilizado pode variar na faixa de nA a mA), tempo de estimulação e repouso (dependendo da aplicação são usados $\mathrm{\mu}$s ou ms). Além disso, na Figura \ref{fig:onb_f6}, pode-se observar que o sinal bifásico é aplicado em geral de forma simétrica/balanceada, ou seja, a carga total (ver Figura \ref{fig:fon_f5}) da fase positiva é igual à carga total negativa. Caso contrário o sinal caracteriza-se como assimétrico/desbalanceado.

\vspace{0.3cm}

\begin{figure*}[h]
    \centering %
    \small %
    \def\svgwidth{0.9\columnwidth}% Código LATEX define o tamanho da figura 
    \import{figs/Fig_c2/}{fon_f5.pdf_tex}
    \caption{Formatos de onda típicos em aplicações de estimulação elétrica (adaptado de  \cite{Robinson2002EletrofisiologiaEletrofisiologico, Popovic2000ControlDisabled}).}
    \label{fig:fon_f5}
\end{figure*}


\begin{figure*}
    \centering %
    \small %
    %\includegraphics[width=\linewidth]{figs/Fig_c2/ogsn.pdf}
    \def\svgwidth{1\columnwidth}% Código LATEX define o tamanho da figura 
    \import{figs/Fig_c2/}{onb_f6.pdf_tex}
    \caption{Ondas bifásicas e alguns parâmetros básicos da estimulação elétrica (adaptado de \cite{Faria2006ImplementacaoMedulares}).}
    \label{fig:onb_f6}
\end{figure*}

Como mencionado, existem diferentes parâmetros de estimulação associados à excitabilidade neuromuscular. Na Tabela \ref{tab:pe1} são identificados os parâmetros de baixo nível, usados na geração do sinal do estímulo elétrico. A amplitude, a largura e a frequência do pulso se relaciona diretamente com a carga total entregada ao músculo, assim relacionadas à geração de maior torque \cite{Popovic2000ControlDisabled}. 

\vspace{0.5cm}

\begin{table}
    \caption{Parâmetros de estimulação.}
    \centering
    \begin{tabular}{|c|c|}
        \hline
        \rowcolor[HTML]{C0C0C0} 
        \textbf{Parâmetro físico} & \textbf{Unidade}                                                                         \\ \hline
        Forma de onda (sinal)     & \begin{tabular}[c]{@{}c@{}}Quadrada, exponencial \\ (Monofásico / Bifásico)\end{tabular} \\ \hline
        Largura de Pulso          & $\mathrm{\mu}$s                                                                                       \\ \hline
        Frequência                & Hz                                                                                       \\ \hline
        Corrente                  & mA                                                                                       \\ \hline
        Carga                     & $\mathrm{\Omega}$      \\ \hline
    \end{tabular}
    \label{tab:pe1}
\end{table}

Como mencionado, para despolarizar artificialmente o nervo e gerar uma contração muscular a partir da eletroestimulação, deve ser fornecida corrente suficiente para gerar na vizinhança da membrana neural potencial elétrico que permita a geração de potencial de ação. Entre os principais parâmetros que definem tal limiar, encontra-se a carga total entregue ao motoneurônio.

\vspace{1cm}
Usando uma forma de onda quadrada monofásica, a carga total se calcula como o produto de corrente aplicada e a largura do pulso. A Figura \ref{fig:qp_f7} ilustra a carga total de um sinal quadrado e a Equação (2.1) mostra o cálculo desta carga (Q) \cite{Robinson2002EletrofisiologiaEletrofisiologico}.

\vspace{0.3cm}
\begin{figure*}[h]
    \centering %
    \small %
    %\includegraphics[width=\linewidth]{figs/Fig_c2/ogsn.pdf}
    \def\svgwidth{0.5\columnwidth}% Código LATEX define o tamanho da figura 
    \import{figs/Fig_c2/}{qp_f7.pdf_tex}
    \caption{Carga total (Q), intensidade aplicada (Ia) e largura de pulso (Lp).}
    \label{fig:qp_f7}
\end{figure*}



Modulando os parâmetros de estimulação na carga total aplicada ao músculo pode-se controlar a intensidade (força) da contração muscular \cite{Maffiuletti2010PhysiologicalStimulation}. Para se gerar uma força persistente, conhecida como uma contração tetânica, uma frequência mínima é necessária. Na aplicação da estimulação elétrica em tecido neuromuscular, várias frequências são recomendadas. Enquanto alguns autores sugeriram frequências entre 20Hz a 80Hz, outros sugerem valores entre 2Hz a 150Hz \cite{Kitchen2003EletroterapiaEvidencias, Delitto1988ElectricalSurgery}.


\subsection{APLICAÇÕES}
Em geral, a estimulação elétrica vem sendo aplicada mundialmente em várias áreas da saúde e esporte. Neste último, por exemplo, é usada para auxiliar o recrutamento de fibras musculares e aprimorar o desempenho do atleta \cite{Garcia2001ElectrostimulationSport}. Para a promoção da saúde, há uma ampla gama de aplicações, como em analgesia para controle da dor em diversas situações clínicas \cite{Sterin1966LosDolor} ou na \acrshort{UTI} para prevenção e tratamento do desuso muscular causado pela imobilidade \cite{Fernando2017ARevisao}. Assim, com o tempo, pesquisas no campo do uso da estimulação elétrica e seus benefícios tem ganhado espaço em diferentes áreas de atuação, aumentando as possibilidades de aplicação deste recurso. Por exemplo, hoje em dia a estimulação elétrica é usada não só em problemas de lesões ou doenças do músculo estriado, mas também em estudos sobre seu mecanismo de ação em patologias neurológicas, assim como uso em musculatura lisa, cardiopatas, portadores de neoplasias, encefalopatias infantis e até pacientes com limitações respiratórias \cite{Paulo2007EfeitosEsqueletico, Sterin1966LosDolor}. Um aspecto importante é que cada aplicação se diferencia e depende dos parâmetros de estimulação, o tempo da sessão e a periodicidade da mesma \cite{Crepon2008Electroterapia.Electroestimulacion, Naki2011IsParameters, Robinson2002EletrofisiologiaEletrofisiologico}.

No contexto desse trabalho, os efeitos buscados com o uso da eletroestimulação são prevenir a atrofia muscular em pacientes com lesão medular e/ou a aparição/aquisição de doenças como a \acrshort{PNMDC}. Para isto, são usados equipamentos eletromédicos de eletroestimulação ou também conhecidos como eletroestimuladores. 
No caso de indivíduos com a excitabilidade neuromuscular muito alterada, como paciente acamado ou imóvel, muitas vezes os eletroestimuladores comerciais não conseguem estimular adequadamente o paciente. Isto deve-se à condição do doente, que como consequência do seu imobilismo, possui uma alta degradação muscular, exigindo que a intensidade e largura do pulso do estímulo elétrico sejam maiores do que em uma pessoa sadia \cite{Paulo2007EfeitosEsqueletico}. Consequentemente, os parâmetros de estimulação não devem ser considerados universais para cada paciente. De fato, a variação de tais parâmetros depende de inúmeras variáveis físicas, incluindo características do eletrodo. 

A identificação de tais parâmetros é em geral realizada pelo terapeuta ou usuário no início do procedimento. Uma vez posicionados os eletrodos, o nível de contração desejada é usualmente estimado visualmente a partir da aplicação de intensidades gradualmente maiores de \acrshort{ES}. 


\subsection{ESTIMULAÇÃO ELÉTRICA E FADIGA MUSCULAR}
% The delivery of electrical stimulation can be customized to reduce fatigue and optimize force output by adjusting the associated stimulation parameters. A full understanding of the settings that govern the stimulation is vital for the safety of the patient and the success of the intervention. Neuromuscular Electrical Stimulation for Skeletal Muscle Function

Como mencionado anteriormente a \acrshort{ES} é usada em inúmeras aplicações hoje em dia. Em aplicações como a reabilitação clínica existe uma limitação no uso da estimulação elétrica, sendo ela a diminuição da eficiência da contração com tendencia ao desenvolvimento de fadiga neuromuscular \cite{DeOliveira2018NeuromuscularAdults}.
Nesse contexto, define-se a fadiga neuromuscular como a incapacidade de realizar ou produzir determinado nível de força de forma repetida no tempo \cite{Ascensao2003FisiologiaPeriferica}.

É importante ressaltar que a diminuição da eficiência da contração se relaciona com mecanismos de origem central e periférico. As causas da fadiga muscular central durante o exercício residem nas regiões corticais e subcorticais. Já a fadiga muscular periférica reside a nível do tecido muscular esquelético \cite{Ascensao2003FisiologiaPeriferica}. 

No anterior contexto, para compensar ou retrasar o aparecimento da fadiga durante a terapia/treinamento deve-se então projetar estrategias como parte do procedimento da aplicação da estimulação elétrica. É por isto que a \acrshort{ES} pode ser personalizada ajustando os parâmetros de estimulação visando reduzir a fadiga e otimizar consequentemente a força produzida \cite{LamA.DoucetB.2012NeuromuscularFunction.}.




\subsection{ELETRODIAGNÓSTICO DE ESTÍMULO}

O eletrodiagnóstico é um exame no qual é aplicada corrente elétrica no conjunto pele-nervo-músculo, para o estudo da excitabilidade neuromuscular, com o intuito de fornecer apoio diagnóstico e oferecer prognósticos para várias lesões ou doenças do sistema neuromuscular \cite{Tejada1997ElectrodiagnosticoEstimulacion}.

O eletrodiagnóstico pode ser realizado por estimulação, detecção ou estímulo-detecção. Ou seja, pela análise da resposta à estimulação elétrica dos nervos ou músculos por meio da pele; pela detecção dos potenciais de ação gerados por um impulso nervoso ou contração muscular por meio de eletromiografia (\acrshort{EMG}); ou também, pelo registro das respostas em presença de potenciais evocados por estímulos somatossensoriais \cite{Fernandes2016StimulusRecovery, Paternostro-Sulga2002ChronaxieDenervation, Kimura2013ElectrodiagnosisMuscle}. 

O eletrodiagnóstico por detecção e por estímulo-detecção são exames que precisam de equipamentos sofisticados para permitir a análise e diagnóstico, devido à natureza do processamento do sinal que implica o procedimento. Por outro lado, o eletrodiagnóstico por estimulação ou, também, eletrodiagnóstico de estímulo, pode ser realizado por meio da inspeção visual, pois a resposta ao estímulo elétrico leva a uma excitação neuromuscular que, consequentemente se traduz em uma contração muscular claramente visível \cite{Naki2011IsParameters, Fernandes2016StimulusRecovery}. 

Neste trabalho o eletrodiagnóstico de estímulo (\acrshort{EDE}) tomasse como referencia para realizar testes de excitabilidade, pois pretende-se automatizar o procedimento relacionado a esse intuito, reduzindo subjetividades decorrentes da inspeção visual. Essa tarefa pode ser realizada, por exemplo, por meio de sensores de movimento para identificar a contração muscular ou perturbação motora (mecânica) no instante em que se aplica um estímulo elétrico e a contração muscular ocorre.

O exame de \acrshort{EDE} está relacionado a três testes: reobase, cronaxia e acomodação \cite{Russo2004AlteracoesEletroestimulacao}. A reobase (\acrshort{RB}) corresponde à mínima corrente de estimulação de duração idealmente infinita capaz de evocar artificialmente uma contração muscular que pode ser percebida visualmente. Os parâmetros de estimulação da \acrshort{RB} correspondem a uma forma de onda quadrada bipolar simétrica com tempo de estimulação ($\mathrm{\alpha}$ na Figura \ref{fig:rca_f8}) de 1s e 2s de tempo repouso ($\mathrm{\theta}$ na \ref{fig:rca_f8}). O valor da corrente de estimulação (amplitude do sinal) por definição é iniciado com 1mA e vai aumentando em passos de 1mA até finalizar o teste, quando ocorre claramente a primeira contração visível, sendo esse o valor da amplitude o resultado do teste \cite{Irnich2010TheOld, Pieber2015OptimizingStudy}. Já a cronaxia (\acrshort{CR}) corresponde ao menor $\mathrm{\alpha}$ necessário para produzir artificialmente uma contração muscular considerando intensidade dobrada em relação à \acrshort{RB}. Os parâmetros de estimulação da \acrshort{CR} correspondem então a uma forma de onda quadrada bipolar simétrica com amplitude igual a duas vezes o resultado obtido previamente no teste de reobase. Define-se um $\mathrm{\theta}$ de 2s e um $\mathrm{\alpha}$ que inicia em 50$\mathrm{\mu}$s e vai aumentando em passos de 50 ou 100$\mathrm{\mu}$s até finalizar o teste quando ocorre claramente a primeira contração visível, sendo tal largura de pulso o resultado do teste \cite{Irnich2010TheOld, Pieber2015OptimizingStudy}. Finalmente, a acomodação (\acrshort{AC}) é definida da mesma forma que a \acrshort{RB}, mas com a diferença de que a forma de onda é exponencial bifásica simétrica. Portanto, a \acrshort{RB} e \acrshort{AC} são expressadas em mA e a \acrshort{CR} em $\mathrm{\mu}$s \cite{Russo2004AlteracoesEletroestimulacao}. A Figura \ref{fig:rca_f8} ilustra as formas de onda com os parâmetros dos testes \acrshort{RB}, \acrshort{CR} e \acrshort{AC}. 

A \acrshort{RB} e a \acrshort{CR} permitem identificar os níveis de excitabilidade muscular (parâmetros de amplitude de corrente e largura de pulso mínima para gerar uma contração, respectivamente), sendo que o resultado do teste de \acrshort{CR} está relacionado com a eficiência do uso de \acrshort{EENM} no tratamento de pacientes acamados e/ou imóveis \cite{Naki2011IsParameters, Martin2014PracticasFisioterapia}. Já a acomodação é a propriedade que tem sistemas neuromusculares íntegros de reagir de forma menos efetiva aos estímulos de crescimento lento (ver forma de onda na Figura \ref{fig:rca_f8}). O valor da amplitude de corrente achado neste teste, permite identificar o nível de comprometimento neuromuscular, possibilitando o diagnóstico de doenças e/ou do estado do complexo motor e sensitivo do paciente \cite{Sabater2012EfeitoRatos}.

Vale relembrar que, para realização do exame \acrshort{EDE}, é necessário contar com um equipamento que permita personalizar e controlar os parâmetros de estimulação, a fim de configurar cada teste.

\begin{figure*}[ht]
    \centering %
    \small %
    %\includegraphics[width=\linewidth]{figs/Fig_c2/ogsn.pdf}
    \def\svgwidth{1\columnwidth}% Código LATEX define o tamanho da figura 
    \import{figs/Fig_c2/}{rca_f8.pdf_tex}
    \caption{Formas de onda dos testes de \acrshort{RB}, \acrshort{CR} e \acrshort{AC}.}
    \label{fig:rca_f8}
\end{figure*}

É de esclarecer que neste trabalho pretende-se usar os testes de reobase e cronaxia para determinar os valores ótimos dos PE para realizar a terapia como evidenciado por Silva et al \cite{Silva2016SafetyStudy.}. Também pretende-se usar o teste de reobase para verificar a hipótese de que pode ser usado como um indicativo da fadiga enquanto se aplica a estimulação elétrica em procedimentos de terapia e/ou treinamento com exercícios isométricos ou resistidos.


%%%% Esta parte tem que rever se vai no texto ou não %%%%%%%%%%%%%%%%%
\section{DISPOSITIVOS DE ESTIMULAÇÃO ELÉTRICA E ELETRODIAGNÓSTICO}
Independentemente do tipo de aplicação da estimulação elétrica, sempre é necessário um dispositivo de eletroestimulação. Assim, um dispositivo de estimulação elétrica ou eletroestimulador, usado tanto para terapia como diagnostico permite administrar, controlar e personalizar a corrente e/ou tensão aplicada no tecido por meio de placas metálicas, eletrodos ou outros acessórios com a finalidade de evocar artificialmente um efeito neural ou uma contração muscular.

Hoje em dia os dispositivos de estimulação, por suas caraterísticas de uso médico-esportivo, se classificam como aparelhos eletromédicos e, como tal, tem de cumprir as normas de proteção e segurança estabelecidas pela lei para este tipo de equipamentos, como no caso das normas técnicas para aparelhos eletromédicos \acrshort{ABNT}\acrshort{NBR} \acrshort{IEC} 60601-1 e \acrshort{ABNT} \acrshort{NBR} \acrshort{IEC} 60601-1-2.
A seguir, são descritos os componentes típicos de um sistema de eletroestimulação. Adicionalmente são descritos os componentes que permitem que o eletroestimulador possa realizar o exame de \acrshort{EDE}.


\subsection{COMPONENTES DE UM SISTEMA TÍPICO DE ESTIMULAÇÃO ELÉTRICA}
Um eletroestimulador é composto geralmente pelos módulos ilustrados no diagrama de blocos da Figura \ref{fig:se_f9}, onde além do módulo estimulador, existem outros módulos como a interface de usuário, módulo de controle e módulo gerador de sinal. Adicionalmente, em alguns casos pode existir o módulo de detecção de movimento, usado para identificar a aplicação do exame de \acrshort{EDE}. 

\begin{figure*}[h]
    \centering %
    \small %
    %\includegraphics[width=\linewidth]{figs/Fig_c2/ogsn.pdf}
    \def\svgwidth{1.3\columnwidth}% Código LATEX define o tamanho da figura 
    \fontsize{10pt}{11pt}\selectfont% or whatever fontsize you like
    \def\svgwidth{4in}
    \import{figs/Fig_c2/}{se_f9.pdf_tex}
    \caption{Diagrama de blocos geral de um dispositivo de estimulação elétrica.}
    \label{fig:se_f9}
\end{figure*}

A seguir são descritos brevemente o funcionamento de cada módulo:

\begin{itemize}
    \item Módulo da interface de usuário (\acrshort{MIU}). Este módulo permite personalizar os parâmetros de estimulação, além de apresentar visualmente diversas informações referentes ao uso do dispositivo. Também, este componente pode ter diversas formas de ser implementado, variando de fabricante a fabricante. Assim, a interface de usuário pode conter botões, telas \acrshort{LCD}, potenciômetros, entre outros.
    
    \item Módulo de controle (\acrshort{MC}). Este componente permite manipular e tratar os valores ingressados na interface de usuário dos parâmetros de estimulação, além de administrar todos os recursos do dispositivo quando o controle é centralizado. Geralmente este módulo é desenvolvido e implementado usando microprocessadores ou microcontroladores. Este módulo pode conter ou fazer a função do módulo gerador de sinal.
    
    \item Módulo gerador de sinal (\acrshort{MGS}). Este componente é o coração do estimulador e é responsável pela formação da onda que comanda o estímulo elétrico. Normalmente o estímulo pode ser monofásico ou bifásico, simétrico ou assimétrico, com ou sem trens de pulso e/ou rampas de subida e decida. O gerador é encarregado do formato da onda e suas componentes como a largura, amplitude e finalmente os tempos de estimulação e repouso. Em resumo, este módulo depende dos parâmetros de estimulação para gerar um sinal de estímulo que geralmente é baseado em tensão.
    
    \item Módulo estimulador (\acrshort{ME}). Este componente é também conhecido como estágio de alta potência, normalmente se relaciona com uma fonte de corrente constante para uma carga variável tendo como referência um sinal de tensão. Neste módulo, o sinal de tensão que vem do módulo gerador comanda uma fonte de corrente que é alimentada por um estágio de potência (fonte de alta tensão), fazendo com que o estímulo tenha um valor de corrente constante independente de mudanças na impedância eletrodo-pele no paciente. 
    
    \item Módulo de detecção de movimento (\acrshort{MDM}). Este módulo é adicional aos componentes típicos de um dispositivo de estimulação. Normalmente os equipamentos de estimulação trabalham em malha aberta, porém, alguns trabalham também malha fechada \cite{Sanches2013SistemaParaplegicos}. A malha fechada pode ser implementada de diversas formas: com sensores de movimento, detecção visual por visão computacional, eletromiografia, entre outros. Nesse contexto, o módulo de detecção de movimento não só conta com sensores que medem ou ajudam a reconhecer uma contração muscular, mas também pode possuir algoritmos que ajudam a tratar os dados dos sensores e finalmente detectar a contração muscular como uma perturbação ou movimento que, em consequência, pode ter uma intensidade medida em força e duração.

\end{itemize}


A partir dessas definições é possível destacar que cada módulo pode ser projetado e implementado de diversas formas. De fato, com relação ao desenvolvimento eletrônico, existem múltiplas soluções e/ou topologias de circuitos eletrônicos para cada módulo, como descrito de forma detalhada no Cap. \ref{sec:cap3}. Nesse momento, é importante ressaltar ainda um aspecto adicional relativo ao estágio de potência. 

Uma contração muscular pode ser artificialmente induzida tanto por fonte de corrente quanto por fonte de tensão conectada a eletrodos. Dessa forma, o projeto do eletroestimulador é baseado na maioria dos casos por estágios de saída controladas por por corrente. 

O dispositivo onde o estágio de saída é controlado por tensão é considerado intrinsecamente seguro para aplicações transcutâneas, pois no instante em que ocorre um erro no eletrodo (como o descolamento de um eletrodo autoadesivo) a densidade de corrente não irá tornar-se suficientemente alta ou perigosa como para induzir queimaduras ou irritações na pele. Em contrapartida há uma dependência da inevitável variação da impedância eletrodo-pele e um estágio de saída controlado por tensão é menos eficiente que a controlada por corrente \cite{Wu2002AApplications}. Os dispositivos com o estágio de saída controlado por corrente são mais indicados para diversas aplicações, visto que regulação da carga fornecida em um cenário de variação da impedância do eletrodo-pele, permite obter, por exemplo, respostas motoras à estimulação mais previsíveis \cite{Wu2002AApplications}. Entretanto, normalmente estimuladores controlados por tensão são caracterizados por circuitos eletrônicos de controle menos complexos de se implementar, em oposição a estimuladores controlados por corrente \cite{Agarwala1986ProgrammableSystem, H.1990PowerStimulator}. Devido sobretudo à precisão no controle do estímulo fornecido, neste trabalho é escolhido o dispositivo de estimulação elétrica com estágio de saída controlado por corrente.